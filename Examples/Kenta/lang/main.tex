\RequirePackage[l2tabu,orthodox]{nag}
\documentclass[a4paper]{article}

%
% my settings
%

% font
\usepackage[utf8]{inputenc}
\usepackage[T1]{fontenc}
\usepackage{textcomp}
\usepackage{lmodern}
\usepackage{exscale} % load this between lmodern and amsmath

% general
\usepackage{graphicx}
\usepackage[unicode]{hyperref}
\usepackage{etoolbox}
\usepackage{xparse}

% math
\usepackage{latexsym}
\usepackage{amssymb,amsmath}
\usepackage{mathtools}
\usepackage{stmaryrd}
\usepackage{bussproofs}

\usepackage[noadjust]{cite}
\usepackage{enumitem}
\usepackage{microtype}


\newcommand{\Pskip}{\mathtt{skip}}
\newcommand{\Pabort}{\mathtt{abort}}
\newcommand{\Passign}[2]{#1\coloneqq #2}
\newcommand{\Pdiscard}[1]{\mathtt{discard}\; #1}
\newcommand{\Pnewassign}[2]{\mathtt{new}\; #1\coloneqq #2}
\newcommand{\Pseq}[2]{#1; #2}
\newcommand{\Pifthenelse}[3]{\mathtt{if}\,
(#1)\,\{#2\}\,\mathtt{else}\,\{#3\}}
\newcommand{\Pobserve}[1]{\mathtt{observe}\,(#1)}
\newcommand{\Ppchoice}[3]{\{#1\}\,[#2]\,\{#3\}}

\newcommand{\Ptrue}{\mathtt{true}}
\newcommand{\Pfalse}{\mathtt{false}}

\newcommand{\Pjudge}[3]{\langle#1\rangle
\;#2\;\langle#3\rangle}

\DeclarePairedDelimiter{\sem}{\llbracket}{\rrbracket}

\begin{document}

Notes on formulation of language

\section{Formulation 1}

Language without allocation/discard of variable.
Variables are fixed in advance.

\begin{align*}
C\Coloneqq {}&
\Pskip
\\
\mid {}&
\Pabort
\\
\mid {}&
\Passign{x}{E}
\\
\mid {}&
\Pseq{C}{C}
\\
\mid {}&
\Pifthenelse{\xi}{C}{C}
\\
\mid {}&
\Pobserve{\xi}
\end{align*}

Here $E$ is an expression and $\xi$ is a predicate.

Probabilistic choice is not necessary if we allow predicates to be `fuzzy'.
$\Ppchoice{C_1}{p}{C_2}$
is the same as $\Pifthenelse{p\cdot \Ptrue
+(1-p)\cdot\Pfalse}{C_1}{C_2}$

We fix variables with types $\Gamma = x_1:t_1, \dotsc, x_n:t_n$,
and call it a context.
Let $C$ be a program defined in the context $\Gamma$.
We interpret $C$ as
an endomap $\sem{C}\colon\sem{\Gamma}\to\sem{\Gamma}$,
where $\sem{\Gamma}=\sem{t_1}\times\dotsb\times\sem{t_n}$.

Normalised semantics is defined when an initial state $\omega$
on $\sem{\Gamma}$ is given.
It is a normalisation of the composite $\sem{C}\circ\omega$.

\section{Formulation 2}

Language with allocation/discard of variables.

\begin{align*}
C\Coloneqq {}&
\Pskip
\\
\mid {}&
\Pabort
\\
\mid {}&
\Pnewassign{x}{E}
\\
\mid {}&
\Pdiscard{x}
\\
\mid {}&
\Passign{x}{E}
\\
\mid {}&
\Pseq{C}{C}
\\
\mid {}&
\Pifthenelse{\xi}{C}{C}
\\
\mid {}&
\Pobserve{\xi}
\end{align*}

We have only ground types $t$.

Contexts are $\Gamma=x_1:t_1, \dotsc, x_n:t_n$.

Typing rules
\begin{prooftree}
\AxiomC{}
\UnaryInfC{$\Pjudge{\Gamma}{\Pskip}{\Gamma}$}
\end{prooftree}

\begin{prooftree}
\AxiomC{}
\UnaryInfC{$\Pjudge{\Gamma}{\Pabort}{\Gamma}$}
\end{prooftree}

\begin{prooftree}
\AxiomC{}
\RightLabel{($E$ is an expression in $\Gamma$ of type $t$)}
\UnaryInfC{$\Pjudge{\Gamma}{\Pnewassign{x}{E}}{x:t, \Gamma}$}
\end{prooftree}

\begin{prooftree}
\AxiomC{}
\UnaryInfC{$\Pjudge{x:t,\Gamma}{\Pdiscard{x}}{\Gamma}$}
\end{prooftree}

\begin{prooftree}
\AxiomC{}
\RightLabel{($x:t\in\Gamma$ and $E$ is an expression in $\Gamma$ of type $t$)}
\UnaryInfC{$\Pjudge{\Gamma}{\Passign{x}{E}}{\Gamma}$}
\end{prooftree}

\begin{prooftree}
\AxiomC{$\Pjudge{\Gamma}{C_1}{\Gamma'}$}
\AxiomC{$\Pjudge{\Gamma'}{C_2}{\Gamma''}$}
\BinaryInfC{$\Pjudge{\Gamma}{\Pseq{C_1}{C_2}}{\Gamma''}$}
\end{prooftree}

\begin{prooftree}
\AxiomC{$\Pjudge{\Gamma}{C_1}{\Gamma'}$}
\AxiomC{$\Pjudge{\Gamma}{C_2}{\Gamma'}$}
\RightLabel{($\xi$ is a predicate in $\Gamma$)}
\BinaryInfC{$\Pjudge{\Gamma}{\Pifthenelse{\xi}{C_1}{C_2}}{\Gamma'}$}
\end{prooftree}

\begin{prooftree}
\AxiomC{}
\RightLabel{($\xi$ is a predicate in $\Gamma$)}
\UnaryInfC{$\Pjudge{\Gamma}{\Pobserve{\xi}}{\Gamma}$}
\end{prooftree}

We interpret
typing judgements $\Pjudge{\Gamma}{C}{\Gamma'}$
as endomaps $\sem{C}\colon\sem{\Gamma}\to\sem{\Gamma'}$.

Normalised semantics is defined for `closed' program
$\Pjudge{}{C}{\Gamma}$.


\end{document}
